% What kind of text document should we build
\documentclass[a4,10pt]{article}


% Include packages we need for different features (the less, the better)

% Clever cross-referencing
\usepackage{cleveref}

% Math
\usepackage{amsmath}

% Algorithms
\usepackage{algorithm}
\usepackage{algpseudocode}
\usepackage{titlesec}
\usepackage{lipsum}% just to generate text for the example
\usepackage{listings} 
\usepackage{url}


\titlespacing\section{0pt}{12pt plus 4pt minus 2pt}{0pt plus 2pt minus 2pt}
\titlespacing\subsection{0pt}{12pt plus 4pt minus 2pt}{0pt plus 2pt minus 2pt}
\titlespacing\subsubsection{0pt}{12pt plus 4pt minus 2pt}{0pt plus 2pt minus 2pt}

% Tikz
\RequirePackage{tikz}
\usetikzlibrary{arrows,shapes,calc,through,intersections,decorations.markings,positioning}

\tikzstyle{every picture}+=[remember picture]

\RequirePackage{pgfplots}




% Set TITLE, AUTHOR and DATE
\title{Finite Element Method Solver Programming(STE 6219)}
\author{Jiaxin Lin(140740)}
\date{\today}
 


\begin{document}



  % Create the main title section
  \maketitle

  \begin{abstract}
   This is the report about the FEM Solver Programming.
   Using GMlib and class TriangleFasets(TriangleSystem) from UiT-Campus Narvik.The report about the finite element method theory,both of regular point set and a random generated point set,compute the matrix and demo of project.
   
  \end{abstract}


  %%%%%%%%%%%%%%%%%%%%%%%%%%%%%%%%%%%%%%
  %%  The main content of the report  %%
  %%%%%%%%%%%%%%%%%%%%%%%%%%%%%%%%%%%%%%
 
  \section{Introduction}
\vspace{10pt}
 \subsection*{GMlib Library}
 \vspace{10pt}
GMlib is free software for use with geometric modelling with parametric in C++.In the project,
Visualization of the drum is using GMlib.The libary was make and update by UIT-Campus Narvik.

   \subsubsection*{Class TriangleFasets}
%\vspace{10pt}

\begin{table}[H]
\centering
\caption{main class}
\vspace{10pt}
\label{main class}
\begin{tabular}{lll}
\cline{1-2}
\multicolumn{1}{|l|}{TSTriangle} & \multicolumn{1}{l|}{\begin{tabular}[c]{@{}l@{}}The triangle class defined by 3 edges,use getVertices to\\ get triangle each vertex and the return value's type is Array\end{tabular}}                                                  &  \\ \cline{1-2}
\multicolumn{1}{|l|}{TSVertex}   & \multicolumn{1}{l|}{\begin{tabular}[c]{@{}l@{}}The vertex class storing 3D position and a normal,each nodes \\ in the drum combine different vertices of triangles.\end{tabular}}                                                      &  \\ \cline{1-2}
\multicolumn{1}{|l|}{TSEdge}     & \multicolumn{1}{l|}{\begin{tabular}[c]{@{}l@{}}The edge class defined by 2 vertices.use getFirstVertex and \\ getLastVertex compare the vertex to get same edges of two triangles\\ and then define the other two nodes.\end{tabular}} &  \\ \cline{1-2}
                                 &                                                                                                                                                                                                                                        
\end{tabular}
\end{table}

In the TriangleSystem have lots of different functions to get something to use in project.
In this report not decrible in detail.
Detailed instructions can be see in the referrence.Just look the inheritance diagram follow the
figure.

    \begin{figure}[H]
      \centering
      \includegraphics[width=1.00\textwidth]{gfx/TriangleFacets.png}

      \caption{Inheritance diagram for TriangleFacets\cite{h}}
      \label{fig:TriangleFacets}
    \end{figure}

  \section{Theoretical Description }

    The finite element method (FEM) is a numerical technique for finding approximate solutions to boundary value problems for partial differential equations.
    \begin{figure}[H]
      \centering
      \includegraphics[width=0.30\textwidth]{gfx/fi.png}
      \includegraphics[width=0.30\textwidth]{gfx/base.png}

      \caption{Finite triangulation of the body\cite{IEEEhowto:kopka}}
      \label{fig:fi}
    \end{figure}

    \begin{figure}[H]
      \centering
      \includegraphics[width=0.70\textwidth]{gfx/commonarea.png}
      \caption{The two Örst Ögures shows the support of the basis function corresponding
to node 1 and node 2 respectively. The last figure shows the intersection
between these supports\cite{IEEEhowto:kopka}}
      \label{fig:fi}
    \end{figure}
Some formular use in the dimension 2.Observe that a ($v_{i}$
, $v_{j}$ ) = 0 unless when Ni and Nj are nodes of the same triangle.

    \begin{figure}[H]
     % \centering
      \includegraphics[width=0.80\textwidth]{gfx/form.png}
      \label{fig:fi}
    \end{figure}

  \section{Random and Regular}
\vspace{10pt}
  \begin{itemize}
    \item Generate random points in drum.
Need two parameter,other can be define by the two paremeter.(radius and number of triangles)
In order to avoid small triangles,have to check two inner points distance longer than s.
($s=\sqrt{\frac{4\pi {r}^{2}}{\sqrt{3}t}}$,t=number of triangles)
\vspace{10pt}

    \item Generate regular points in drum.
Need three parameter begin radius,number of rings,begin points.
From the first ring to increase,add one more ring in that ring's points multiply 2.
  \end{itemize}

  

  \section{Compute Matrix}\vspace{5pt}
  \subsection*{Outside the diagonal}\vspace{5pt}
First compute the stiff matrix.Find three vector to use in formular.So Need to return three values.

    \begin{figure}[H]
      \centering
      \includegraphics[width=0.70\textwidth]{gfx/stiff.png}

      \caption{Computing the stiffness matrix\cite{IEEEhowto:kopka}}
      \label{fig:TriangleFacets}
    \end{figure}



    \begin{itemize}
      \item Formula for three vector\\
      $\overrightarrow{d}=p_{1}-p_{0}$\vspace{5pt}\\
$\overrightarrow{{a}_{1}}=p_{2}-p_{0}$\vspace{5pt}\\
$\overrightarrow{{a}_{2}}=p_{3}-p_{0}$\vspace{5pt}
\item Computing Stiffness matrix elements outside the diagonal.Use the three vector to combine the below formula .
\begin{lstlisting}[ language=C++] 
  if(sameEdge!=nullptr){
      auto getVector=findVector(sameEdge);
      double dd=1/(getVector[0]*getVector[0]);
      //first trangle
      double area1=std::abs(getVector[0]^getVector[1]);
      double dh1=dd*(getVector[1]*getVector[0]);
      double h1=dd*area1*area1;
      //sencond triangle
      double area2=std::abs(getVector[2]^getVector[0]);
      double dh2=dd*(getVector[2]*getVector[0]);
      double h2=(dd*area2*area2);
      //computing Amatrix[i][j]

  _Amatrix[i][j]=_Amatrix[j][i]=(((dh1*(1-dh1)/h1)-dd)*
((area1)/2)+((dh2*(1-dh2)/h2)-dd)*((area2)/2));

                   }
\end{lstlisting} 
\end{itemize}
\subsection*{Diagonal}

 Computing stiffness matrix diagonal elements
\begin{figure}[H]
      \centering
      \includegraphics[width=0.70\textwidth]{gfx/di.png}

      \caption{Computing the stiffness matrix diagonal elements\cite{IEEEhowto:kopka}}
      \label{fig:TriangleFacets}
    \end{figure}
\begin{itemize}
\item The values on the diagonal of the stiffness matrix is where the element is computed against itself.
\\
$\overrightarrow{d}_{1}=p_{2}-p_{0}$\vspace{5pt}\\
$\overrightarrow{{d}_{2}}=p_{1}-p_{0}$\vspace{5pt}\\
$\overrightarrow{{d}_{3}}=p_{2}-p_{1}$\vspace{5pt}\\
 
\begin{lstlisting}[ language=C++] 
 _Amatrix[i][i]+=((Dvector[2]*Dvector[2])/
(2*std::abs(Dvector[0]^Dvector[1])));
                   
\end{lstlisting} 
\end{itemize}
\subsection*{Vector B}
\begin{itemize}
\item Computing Vector B
\begin{lstlisting}[ language=C++] 
auto Bvector=findVectorD(_nodes[i],t[j]);
_b[i]+=((std::abs(Bvector[0]^Bvector[1]))/6);
\end{lstlisting} 
    \end{itemize}

  \section{Conclusion}
    In conclusion,this project is  based on the lectures in the course: partial differential equations and FEM.
Througe this project understand the fem better.Practise better than theory.Moreover,with use the 
GMlib libary and see the code form the teacher.Students will make progress in C++ and math.
  
% Include the bibliography
\begin{thebibliography}{1}


\bibitem{h}
\url{http://episteme.hin.no/dox/GMlib2/classGMlib_1_1TriangleFacets.html}

\bibitem{IEEEhowto:kopka}
Dag~Lukkassen, \emph{A short introduction to Sobolev-spaces and applications for engineering students},\hskip 1em plus
  0.5em minus 0.4em\relax November, 2004.
\end{thebibliography}
\end{document}
